% Created 2024-09-08 dim. 18:19
% Intended LaTeX compiler: pdflatex
\documentclass{article}
\usepackage[utf8]{inputenc}
\usepackage[T1]{fontenc}
\usepackage{graphicx}
\usepackage{longtable}
\usepackage{wrapfig}
\usepackage{rotating}
\usepackage[normalem]{ulem}
\usepackage{amsmath}
\usepackage{amssymb}
\usepackage{capt-of}
\usepackage{hyperref}
\usepackage[french]{babel}
\usepackage{cours}
\usepackage{pmboxdraw}
\setcounter{secnumdepth}{2}
\author{Giuseppe Lipari et toute l'équipe pédagogique}
\date{\today}
\title{PDS how to}
\hypersetup{
 pdfauthor={Giuseppe Lipari et toute l'équipe pédagogique},
 pdftitle={PDS how to},
 pdfkeywords={},
 pdfsubject={},
 pdfcreator={Emacs 27.1 (Org mode 9.5.5)}, 
 pdflang={French}}
\begin{document}

\maketitle

\section{Comment tester vos réponses}
\label{sec:org3677453}

Pour compiler les tests, saisissez la commande 

\lstset{language=sh,label= ,caption= ,captionpos=b,numbers=none}
\begin{lstlisting}
make test
\end{lstlisting}

Si vous voulez lancer les tests, saisissez la commande :
\lstset{language=sh,label= ,caption= ,captionpos=b,numbers=none}
\begin{lstlisting}
make runtests
\end{lstlisting}

Si vous voulez simplement lancer un sous-ensemble des tests, vous
devez d'abord vous déplacer dans le repertoire \texttt{test}, ensuite lancer
la commande \texttt{test\_ex1} pour les tests de l'exercice 1:

\lstset{language=sh,label= ,caption= ,captionpos=b,numbers=none}
\begin{lstlisting}
cd test && ./test_ex1
\end{lstlisting}

Si vous désirez afficher la liste des tests sans les lancer, tapez :

\lstset{language=sh,label= ,caption= ,captionpos=b,numbers=none}
\begin{lstlisting}
./test_ex1 -l -v
\end{lstlisting}

Si vous souhaitez lancer seulement les tests relatifs à la question 2
de l'exercice 1, tapez :

\lstset{language=sh,label= ,caption= ,captionpos=b,numbers=none}
\begin{lstlisting}
./test_ex1 ex1.q2 
\end{lstlisting}

Des commandes similaires sont disponibles pour l'exercice 2.
\end{document}